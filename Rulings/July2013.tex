\section*{Acess to Investigation Minutes}
\begin{center}
\rule{\textwidth}{3.6pt}\\[\baselineskip] % Thick horizontal linee
\begin{Huge}
\textbf{Speaker's Ruling}
\end{Huge}

\rule{\textwidth}{3.6pt}\\[\baselineskip] % Thick horizontal line



\vspace*{2\baselineskip} % Whitespace Between Title and Discriptive Title

{\large  \textbf{Access to Investigation Minutes}}
\end{center}



Dear SOGS Council,
I have received a multi-part question from a member, but as the questions relate to a single issue, I have addressed them with one ruling. I have divided my ruling into five parts:
\begin{itemize}
\item The Question
\item Preamble
\item Ruling (parts 1-4)
\item Recommendations
\item Relevant Bylaws
\end{itemize}

\begin{multicols}{2}

\begin{longenum}[ label*=\arabic*., align=left]
\item \textbf{The Question}

\begin{itemize}
\item[A)] Does Bylaw 14 --- specifically sections 14.1 and 14.4 (see below) --- give a member of the Society to view the minutes of the recent Ad Hoc Investigative Committee (within the restrictions set out in 14.5)?
\item[B)]  Were the Non-Disclosure Agreements (NDA) signed by the members of the Investigative Committee and complainants out of order?
\end{itemize}

\item \textbf{The Bylaws in Question}

14.1 All minutes of Council or the Executive, and,  where compiled,  of committees shall be available to all interested parties, with the exceptions noted below.
14.4 Confidential minutes may be viewed only by full and associate members of the Society.

\item \textbf{Preamble}

In November 2012, the SOGS Executive received a complaint about the conduct of a member of the Society. In response, an ad hoc committee was struck --- under the guidance of the Speaker --- to investigate the complaint, determine its validity, and recommend discipline, if appropriate. As SOGS has no pocedures in place to deal with such complaints of discipline, the process was guided by the process laid out in Robert's Rules of Order Newly Revised, 11th Edition (RONR), which stresses confidentiality at all points in the process. To this end, the members of the committee and the individuals who brought forward complaints all signed NDAs.

As the preamble to Bylaw 14 indicates, the Society is committed to allowing open access to information. As such, even confidential minutes of the Society are made available to members to review. In this case, however, we have minutes which are rendered confidential by a legal document. This tension between the principles of Bylaw 14 and the NDAs signed by those involved in the investigation is the core issue here.

\item \textbf{The Ruling}
\begin{longenum}
\item \textit{Summary}

Because the NDAs signed by those involved in the investigation are legal documents, and because the authority of the law exceeds that of the governing documents of SOGS in any situation in which they conflict (\#2 below), I rule that members of the Society do not have a right to view the minutes of the investigative committee, as the NDAs in place create a legal requirement that they not be disseminated.
I am not, however, able to say for certain that the NDAs are absolutely binding or advise on their legal status. An answer to the question of the exact legal scope of the NDAs, and conditions under which the information protected by these agreements\footnote{i.e., the minutes of the committee and details of the investigation}  is a matter of legal opinion which is beyond both the authority and capabilities of the Speaker (\#3). As such, I recommend that the President seek advice from SOGS legal counsel on this matter.
Further, as I can find no language in the Bylaws which would serve to prohibit the use of a Non-Disclosure Agreement to protect certain sensitive information, and due to the importance of being able to protect the rights of both the accused and complainants during any investigation, I rule that the NDAs signed in this case were not out of order (\#4).
Finally, as the Bylaws are currently silent on the question of how the Society is to determine the truth of any allegations against its members, or how disciplinary actions are to be determined if required, it is my recommendation than an ad hoc committee be struck to review bylaw 19 and draft a policy for investigations and discipline within SOGS, so that we have a clear guide for how to proceed in the future, should any additional complaints of misconduct arise.

\item \textit{Law and Bylaws}

Bylaw 14 makes it clear that all members have a prima facie right to view all minutes from Society meetings, including minutes otherwise considered confidential, but it does not follow from this prima facie right that there are no circumstances under which minutes may be off-limits to a member.
While our Bylaws are one of the governing documents of the Society which, along with the Constitution and Policy Document, provide the general structure for the operation of the Society, their authority is not absolute. While the Constitution is the highest authority among the Society's governing documents, any legal requirements ? whether federal, provincial, or municipal ? always take precedence over the Society's rules\footnote{Cf. RONR (11th ed.), p. 3-4: ``[T]he actions of any deliberative body are also subject to applicable procedural rules prescribed by local, state, or national law and would be null and void if in violation of such law.''}.  Thus, if the Constitution/Bylaws make one demand while the law makes another, the Society must always take the course of action prescribed by law.
The current case is one in which exactly such a tension appears. The bylaws require that Society members be given access to the minutes, but the NDAs require that they be kept confidential, and not subject to such broad access.

\item \textit{Legal Status of the NDAs}

 The above is written on the assumption that the NDAs create a legal obligation for the Society which prohibits dissemination of the details of the investigation. Should the NDAs create no such obligation, bur rather, for example, only an obligation that the minutes not be posted publicly, then there is no tension between the legal requirement created by the NDAs and the rights granted to members under bylaw 14, and all members of the Society would have the right to view these minutes\footnote{Under the conditions listed in Bylaw 14.5}.  Advising on whether or not this is the case is outside the purview of the Speaker, and would require an opinion from legal counsel.
 
\item \textit{Appropriateness of NDAs}

While Bylaw 14 allows all members of the Society to view even confidential minutes of meetings, this does not mean that an NDA which has the effect of making minutes inaccessible would thereby be out of order. There currently exists no language in the Bylaws or Constitution which prohibits the use of NDAs to protect sensitive information.\footnote{It is important to note that this does not constitute proof that the NDAs are not out of order, but merely evidence in favour of their not being out of order.}  Further, there exist other methods of keeping sensitive proceedings confidential which are used frequently by the Society and are not considered to be out of order or a violation of Bylaw 14, including moving a session in camera. Thus, due to the combination of the lack of language prohibiting their use, and the permissibility of functionally similar mechanisms in Society proceedings leads me to conclude that the use of NDAs in this case was not impermissible.
Separate from this question is the question of whether or not having those involved in the investigation sign NDAs was wise, or the best course of action. Certainly there is good reason to protect the identities and reputations of the accused, the complainants and even the committee members in any such investigation and to have a degree of confidentiality in such cases. NDAs are only one method of doing this, however, and the need for privacy must also be balanced against the Society's general commitment to open access to information. How this balance is to be struck, however, is a foundational question, not an interpretive one, and thus beyond the scope of a Speaker's ruling. How to proceed in such situations in the future must be left to Council.
\end{longenum}
\item \textbf{Recommendations}
As mentioned in section III, part 3, the Speaker is unable to comment on the full legal scope of the NDAs. As a result, I recommend that the President discuss these NDAs with legal counsel to determine exactly how they bind the actions of the Society.
Further, as mentioned in section II and Section III, part 4, the Society's governing documents do not currently lay out any clear procedures for dealing with complaints of misconduct by a member, or determining appropriate discipline. Bylaw 19 discusses what sorts of discipline may be meted out, but not how to go about determining if discipline is warranted in the first place. To resolve this issue I suggest that an ad hoc committee be struck to develop more thorough procedures for the Society to investigate and address member misconduct, and deliver a report containing their recommendations to the Bylaws and Constitution Committee and the Policy committee for vetting and recommendation to Council. During this process, particular attention should be paid to the importance of maintaining confidentiality through the process ? a principle highlighted in the recent changes to the Society's Conflict of Interest policies ? and to balancing this against the Society's commitments to openness.

\item \textbf{Relevant Bylaws}

Bylaw 14: Disclosure of Information
14.1 All minutes of Council or the Executive, and,  where compiled,  of committees shall be available to all interested parties, with the exceptions noted below.
14.4 Confidential minutes may be viewed only by full and associate members of the Society.

\end{longenum}
\end{multicols}


\noindent
Respectfully yours, \newline
\noindent
Christopher Shirreff \newline
\indent
Speaker, \newline 
\indent
Society of Graduate Students \newline
\indent
sogs.speaker@uwo.ca \newline

