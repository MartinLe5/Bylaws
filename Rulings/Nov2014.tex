\setcounter{section}{0}

\begin{center}
\rule{\textwidth}{3.6pt}\\[\baselineskip] % Thick horizontal linee
\begin{Huge}
\textbf{Speaker's Ruling}
\end{Huge}

\rule{\textwidth}{3.6pt}\\[\baselineskip] % Thick horizontal line



\vspace*{2\baselineskip} % Whitespace Between Title and Discriptive Title
\end{center}
\section*{Alternate Councillors for Unassigned Council Seats}


Dear Councillors,
	A member of the Society has asked me the following question: we allow Alternate Councillors to attend meetings when their regular department representative is unavailable. Do the Bylaws allow for an individual to serve as an Alternate for a vacant department seat? My ruling follows.
	
\begin{multicols}{2}
	The Bylaws do not clearly either allow or prohibit this practice. However, it is not practically possibly to abstain from a decision on this matter, as an abstention wold practically amount to banning the practice. In absence of clear guidance from the bylaws, I must turn to other principles to decide the matter. In particular, the Society and its policies are structured around encouraging participation and representation wherever possible. In light of this, I have determined that prohibiting interested individuals from filling in for a vacant Council seat ? absent a policy explicitly banning it ? runs contrary to the foundational principles of SOGS.
	As such, it is my ruling that Councillors may serve as Alternate Councillors for a seat that is currently vacant. The typical rules that apply to ordinary Alternate Councillors apply in this case as well (i.e., they must be a member of the department they are alternating into, and must submit an alternate form bearing their name/student number/departmental affiliation at the start of the meeting), and count as present for the purposes of quorum and departmental grants.
	Finally, as I have stressed, this ruling is not based on an interpretation of any particular bylaw since, as stated above, there is nothing in the bylaws which settles this matter either way, but is rather based on my interpretation of the broader principles which govern the Society. As such, I recommend that the BCC consider this issue and draft a proposed amendment to the Bylaws which will either explicitly allow or prohibit this practice.
\end{multicols}




\noindent
Respectfully yours, \newline
\noindent
Christopher Shirreff \newline
\indent
Speaker, \newline 
\indent
Society of Graduate Students \newline
\indent
sogs.speaker@uwo.ca \newline

