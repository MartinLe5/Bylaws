\section{Procurement}
\begin{longenum}[label*=\thesection.\arabic*., align=left]
\item \textbf{Purpose} \newline
The purpose of this policy is:
	\index{Finance!Procurement|(}
	\index{Procurement|(}
\begin{longenum}[label*=\arabic*., align=left]
		 
\item To ensure that the Society's process in determining the use of funds for the purchase of goods and services is conducted in a publicly accountable manner to the benefit of the membership;

\item To outline the role and responsibilities in facilitating the expedient purchase of goods and services necessary to support the goals and objectives of the Society; and

\item To provide the framework to seek competitive prices in a manner that is in accordance with the resolutions from council, reflect the values of the Society as a not-for-profit organization and promote a fair and appropriate tendering process for all interested suppliers.

\index{Council!Values}

\end{longenum}

\item \textbf{Definitions}

\begin{longenum} [label*=\arabic*., align=left]
	
\item \textbf{Requisitioner}: The Society of Graduate Students.

\item \textbf{Requisitioning Committee:} The committee which prepares a procurement process and reviews bidder submissions as appropriate. Certain contracts (outlined below) require a particular Society Committee to serve as the Requisitioning Committee.
\end{longenum}

\item \textbf{Scope} \newline
This policy applies to the purchase of:
\begin{longenum} [label*=\arabic*., align=left]

	\item Goods and services estimated in excess of \$5,000 excluding taxes and/or shipping costs;
	\item Goods and services of any price as determined at the discretion of council; 
	\item Goods and services identified in a competitive bid from a supplier;
	\item Goods and services requiring a "request for quotation" or "tender" as deemed necessary by government, university or contractual obligation.
	\item The extension or renewal of any exciting contracts for goods as described above unless explicitly excluded in this policy.
	
	\item Existing Contracts: \newline
	New bids are not required for the acquisition of goods and services covered by an existing Society contract. Should a Requisitioner choose not to use a supplier already under contract, the appropriate bid process must be followed.
	
	\item Exclusions: \newline
	Goods purchased for external resale; and
	Goods and services valued up to and including \$5,000 not addressed by the inclusion rules above.

\end{longenum}

\item \textbf{Competitive Bid Requirements}
\begin{longenum} [label*=\arabic*., align=left]
	\item It is the responsibility of the Requisitioner to ensure all purchases requiring competitive bids are compliant with the following:
		\begin{longenum} [label*=\arabic*., align=left]
	\item Prices must be sought from at least three (3) sources;
	\item Wherever possible efforts should be made to seek at least one (1) bid from a non-profit, not-for-profit or cooperative vendor;
	\item Price quotations must be obtained using the method appropriate to the complexity and estimated cost of the good or service under consideration as described below.
\end{longenum}	
\end{longenum}

\item \textbf{Sole Source}
\begin{longenum} [label*=\arabic*., align=left]
	\item The competitive bid requirement is waived for the purchase of those goods and services only available from one (1) lone supplier. An example of a sole source contract is the London Transit Commission Student Bus Pass agreement.
	\index{Purocurement!Sole Source}
	\index{Bus Pass}
\end{longenum}

\item \textbf{Price Quotation}

\begin{longenum} [label*=\arabic*., align=left]
		\item For purchases of \$5,001-\$50,000, price quotations must be solicited by the Requisitioner by phone, fax or e-mail. Price quotations must be documented (including award justification) by the Requisitioning Committee or the Executive, as appropriate. The Requisitioner may contact the suppliers directly or enlist the assistance of the University of Western Ontario's Purchasing Unit.
		
\end{longenum}

\item \textbf{Request for Quotation (RFQ)}

\begin{longenum} [label*=\arabic*., align=left]
		\item  The Requisitioner is responsible for providing formal Request for Quotation documents to all suppliers bidding on purchases as part of the procurement process. A list of suppliers invited to participate in a particular bid shall be maintained by the Chairperson and Official Liaison of the Requisitioning Committee responsible for the request. Public calls must be advertised using the Society's website as well as one (1) or more standard advertising mediums as appropriate. Suppliers' responses must be sealed and forwarded through the Society's Office to the Requisitioner, with attention to the Society's President and Official Liaison to the Requisitioning Committee.
		
	\item	Stamped with the date and time of receipt at the Society's office, bid documents are opened only after the deadline has expired (according to the process outlined in Opening of Bids). The Requisitioning Committee evaluates the tenders and selects the bid for recommendation/approval that best meets the selection criteria. Once a final decision is reached, the Award Committee forwards a formal award notice to all bidding suppliers.
		
\end{longenum}

\item \textbf{Request for Proposal (RFP)}

\begin{longenum} [label*=\arabic*., align=left]
		\item All RFQ and RFP documents must include a detailed description of:
	\begin{longenum} [label*=\arabic*., align=left]	
		\item Goods or services to be purchased; 
	\item 	Deadline date and time;
	\item	E-mail, address and phone number of the Society; 
	\item 	Terms and conditions of the bid; and
	\item 	Terms and conditions of subsequent purchase and payment.
		
\end{longenum}
\item	In any situations where the Requisitioning Committee has agreed to mandatory or rated criteria that will be used to evaluate submissions, these criteria must be clearly outlined including weight of each criterion. Mandatory criteria should be kept to a minimum in order to ensure that no bid is unnecessarily disqualified.
\end{longenum}
\item \textbf{Documents}

\begin{longenum} [label*=\arabic*., align=left]
		\item Only the Society's President and the Official Liaison to the Requisitioning Committee have the authority to place advertisements for public calls for pricing requests.
		
		\item All communications with vendors/suppliers must be conducted by the President and Official Liaison of the Requisitioning Committee unless otherwise stated in this policy.
\end{longenum}

\item \textbf{Advertising of Pricing Requests}

\begin{longenum} [label*=\arabic*., align=left]
		\item Upon receipt by the Society's office, bids are stamped with the date and time. Responses received after the deadline are documented as "late" and either refused or returned to the bidder.
		
		\item In the case where a bid is received before the closing date, the bid shall be stored securely until such time as the bids are to be opened. Under no circumstance shall a tender be opened before the closing date.
\end{longenum}

\item \textbf\textit{}{Receipt of Bids}

\begin{longenum} [label*=\arabic*., align=left]
		\item Any bids are opened in the presence of at least the committee chair or a designate, the Society's President, the Official Liaison to the Requisitioning Committee, the Speaker, and one (1) member of the full-time office staff. At this meeting, the contents of each bid are presented. No further changes to the bids shall be entertained, unless specifically requested by the Requisitioning Committee.
		
		\item All tenders shall be photocopied, and the originals shall remain under lock in the tender box. The President and Official Liaison for the Requisitioning Committee shall summarize bid information and submit it to the Requisitioning Committee. A member of the Requisitioning Committee may elect to request consulting the complete bids by the full committee.
		
		\item Any information on a bid or procurement process is considered privileged and confidential from when the bids are opened until the contract is awarded to one (1) of the bidders. Any Society Executive, staffperson or committee member who is privy to bid information during the procurement process shall sign a non-disclosure agreement for the duration of the process until the contract is executed and signed.
\end{longenum}


\item \textbf{Evaluation and Award}

\begin{longenum} [label*=\arabic*., align=left]
		\item All quotations and requests are created and reviewed by a committee consisting of at least one representative from the Executive, members of the Requisitioning Committee, as well as one (1) non-voting member of the Society's office staff retained to advise on historical precedents. The Requisitioning Committee reviews all responses and files its evaluation and final recommendation with the Executive and Speaker. 
		
		\item Members of the Requisitioning Committee should familiarize themselves with the Society's Constitution, Bylaws and Policies, with special attention to the relevant sections on Conflict of Interest.
		\item All qualifying bids shall be evaluated. All bids and the rationale of the Requisitioning Committee for accepting a bid shall be documented and retained in the Society's office for six (6) years.
		
\end{longenum}

\item \textbf{Requisitioning Committee}

\begin{longenum} [label*=\arabic*., align=left]
		\item No person involved in the tendering process shall solicit or accept gratuities, favours or anything of non-informational value from bidders or potential bidders during the tendering process.
		
		\item Any person involved in the tendering process who knowingly and deliberately, or who reasonably ought to have known, violates the above shall be removed from the tendering process.
		
		\item Any bidder or potential bidder who knowingly and deliberately offers gratuities, favours or anything of non-informational value to those involved in the tendering process shall be subject to having their bid disqualified at the discretion of the Requisitioning Committee. This restriction must be communicated to all potential bidders as part of the RFQ/RFP process.
\end{longenum}


\item \textbf{Vendor Protests}
\begin{longenum} [label*=\arabic*., align=left]


\item The Society is responsible for maintaining a bidding process that is fair and equitable to all interested parties. If a supplier believes the bidding process has been compromised, the bidder is invited to contact the Society for possible investigation according to the parameters outlined below:

\item Suppliers must submit a written protest within five (5) days of learning information applicable to the protest.

\item Once the grievance is filed, the Requisitioner, the Executive, and the Ombudsperson shall be notified. Collectively, these parties decide if the procurement should continue.

\item If the contract is not yet awarded, it may be held until the review is complete, except in circumstances in which delaying awarding the contract would substantially damage the Society and/or its membership.

\item For contracts already awarded, the Requisitioner notifies the supplier awarded the bid and if deemed possible and necessary, may be asked to stop or hold the purchase until the review is complete. When appropriate or necessary, legal counsel is consulted.

\item Before making a determination on a vendor protest, the Executive, the Speaker, the Ombudsperson, the chair of the Requisitioning Committee, and if necessary legal counsel, shall engage in a consultative process to review the available information and evidence. After reviewing the information gathered, the President shall issue a written determination to all involved parties.
\end{longenum}

\item \textbf{Requirements for Particular Tenders}

\begin{longenum} [label*=\arabic*., align=left]
		\item  \textbf{Accountant/Auditor}
		\begin{longenum} [label*=\arabic*., align=left]
			\item The Finance Committee shall be the Requisitioning Committee for the Accountant/Auditor contract. The Accountant/Auditor tender must always be conducted through an RFP process. The length of term for the contract shall be for four (4) years with the possibility of an extension for one (1) additional year.
			\item Upon evaluation of the bids, the Finance Committee shall compile a report, recommendation, and attendant motion to be submitted to Council for approval.
			\item The report shall include: the recommendation and rationale of the committee, as well as the comparisons between the recommended bids and at least two (2) other bids. The recommendation of the committee is not final, and Council may amend the motion to select one of the other bidders.
			\item When considering the quotation from accountants, the incumbent has an advantage on price, the first audit requires more time than subsequent audits. When selecting the accountant, it is important to consider experience with organizations similar to the Society: not-for-profit organizations, memberships of several thousand, a budget in excess of one million dollars.
			\item During the final calendar year of the Accountant/Auditor contract, the committee shall begin preparations for a tendering process for the next contract.
			\item The bid requests shall be sent out no later than the first week of October, with a deadline of November 14th, or the business day immediately preceding.
			\item The Finance Committee shall present its recommendation to Council at the
March meeting.
\end{longenum}

\item \textbf{Health Plan}

		\begin{longenum} [label*=\arabic*., align=left]
			\item The Health Plan Committee shall be the Requisitioning Committee for the health plan contract. The health plan tender must always be conducted through an RFP process.
		\item 	The Society shall not enter into a health plan contract longer than three (3) years.
		\item 	At least one (1) bid must be sought from a not-for-profit or cooperative provider.
		\item Upon evaluation of the bids, the Health Plan Committee shall compile a report to be presented to Council.
		\item The  report  shall  include  an  anonymized  shortlist  of  bids  and  a  comparison  of  the  merits  and  drawbacks  of  each  bid  on  the  shortlist.  The  shortlist  must  include  bids  from  at  least  two  vendors  and  be  comprised  of  40\%  of  the  total  number  of  bids  to  a  maximum  of  five.
		\item During the final calendar year of the health plan contract, the committee shall begin preparations for a tendering process for the next health plan contract.
		\item 	The bid requests shall be sent out no later than the first week of January, with a deadline the last business day of January.
		\item 	The Health Plan Committee shall present its recommendation to Council at the
March meeting.
	\end{longenum}	
	
\item \textbf{Grad Club}

\begin{longenum} [label*=\arabic*., align=left]
	\item The Grad Club Committee shall be the Requisitioning Committee for the Grad Club.
\end{longenum}	

\item \textbf{Hand Book}
\begin{longenum} [label*=\arabic*., align=left]
	\item The Orientation and Social Committee shall be the Requisitioning Committee for the handbook printing contract.
	
	\item The handbook procurement must always be conducted through an RFQ process.
	
	\item At least one (1) bid must be sought from a not-for-profit or cooperative source.
	
	\item The bid requests shall be sent out no later than the third week of January, with a deadline of February 8th, or the business day immediately preceding.
	
	\item The bid should take into consideration time for the handbook editor to access the printer's in-house graphic designer, to allow formatting work on the handbook.
\end{longenum}
\end{longenum}

\end{longenum}
	\index{Finance!Procurement|)}
	\index{Procurement|)}