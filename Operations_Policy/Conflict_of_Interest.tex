\section{Conflict of Interest}


\begin{longenum}[ label*=\arabic*., align=left]
\item  \textbf{Governing Bylaws}

	\begin{longenum}[ label*=\arabic*., align=left]
		\item The Society's Bylaws, Section 11 (Conflict of Interest) shall be used to determine whether or not a Conflict of Interest exists
	\end{longenum}
\item \textbf{Procedure for Resolution:}
	\begin{longenum}[ label*=\arabic*., align=left]
		\item An individual may recognize one's own conflict and follow the procedure outlined in Bylaw 11.2 to resolve a conflict. \textbf{At any point in the following procedural outlines and policies, a person may acknowledge their own conflict and seek steps to resolve it privately, in accordance with the Bylaws.} This should always be encouraged in the interests of saving the Society unnecessary time and energy expended resolving its internal difficulties.
        \index{Conflict of Interest!Informal Resolution}
        \item An individual must first inform the person perceived to be in a Conflict of Interest of their conflict, and must do so clearly, substantively and in writing (electronic media is acceptable). \textit{Vague allegations, generalized concerns and innuendo are not substantive claims.}
        \item Failing to resolve the matter privately and discretely may result in an escalation to the conflicted person's supervisor, in whatever capacity they may be supervised: a committee chairperson, a committee's official liaison, the Society's President, or the Society's Speaker (in order of preference, where applicable). Third parties external to the Society are not considered part of any supervisory chain, and may not be employed.
		\item Failing resolution through a supervisory intervention, the concerned party may do one of two things, depending upon the time sensitivity of the issue (to be determined by the Speaker):
		\begin{longenum}[ label*=\arabic*., align=left]
			\item If the issue is not time-sensitive, the individual may draft a motion for Council and seek resolution in that forum.This is the preferred method of resolution at this level of escalation.
            \item If the issue is of a time-sensitive nature, the individual may seek a n ad hoc tribunal proceeding through the Society's Speaker (see item 3, Tribunal Proceedings, below).
            
		\end{longenum}
		\item The only bodies capable of imposing a decision on a person perceived to be in a Conflict of Interest are the Society's Council (2.2.3.1) and the ad hoc tribunal formed by the Speaker (2.2.3.2). All other methods of resolution must come in the form of recommendations to the ostensibly conflicted individual, and to which all parties involved must agree.
        \index{Conflict of Interest!Role of the Speaker}
        \index{Speaker, The!Conflict of Interest Resolution}
        \item At each stage of attempted resolution, a reasonable amount of time must be allowed to respond.
	\end{longenum}
	\item \textbf{Tribunal Proceedings}
	\begin{longenum}[ label*=\arabic*., align=left]
		\item In the event a perceived Conflict of Interest remains unresolved and is of a time-sensitive nature that cannot wait until a Council meeting, the Speaker may be asked to call an ad hoc tribunal to impose a judgement upon the conflict situation in question.
                \index{Conflict of Interest!Time-Sensitive Matter}
        \item The ad hoc tribunal shall only be called if the complaint is received in writing to the Society's office, and addressed to the Speaker.
                 \index{Conflict of Interest!Time Insensitive Matter}

        \item A tribunal shall consist of four (4) Society members in good standing: the Speaker (non-voting), and three other SOGS members representing three different faculties. These members shall be selected by the Speaker, and should themselves be free of any reasonable apprehension of bias toward both the complainant and defendant.
                \index{Conflict of Interest!Tribunal!Composition}

        \item The tribunal shall assemble within seven business days, barring unforeseen circumstances, and shall offer a ruling based on the most complete testimony of the individuals involved. Follow up queries are permitted for clarification, and at all points the Speaker shall provide guidance on the interpretation of the Society's Bylaws to assure that the minimum standards for a Conflict of Interest are met (most notably, that of demonstrable personal or financial gain).
                \index{Conflict of Interest!Tribunal!Procedure}
                \index{Conflict of Interest!Tribunal!Role of Speaker}


        \item At all points, the tribunal must strive for unanimity in its resolutions.
	\end{longenum}
	\item \textbf{Possible Resolutions}
	\begin{longenum}[ label*=\arabic*., align=left]
		\item Any person, supervisor, Council, or ad hoc tribunal may determine that no conflict exists, and thus dismiss the charge. Reasons should be documented as best as possible, with resolutions ready to be provided in the event the matter is escalated.
     \index{Conflict of Interest!Resolutions}        
        \item Any person or supervisor (as described in 2.2) may determine that a conflict does exist, and thus may recommend the following, in order of desirability:
		\begin{longenum}[ label*=\arabic*., align=left]
			\item The conflicted individual be asked to cease participating in the situation generating the conflict, either by noted abstention during voting on the issue, by leaving the room during a committee meeting in which the issue arises, or some other similarly appropriate and generally benign measure of resolution.  
            \item The conflicted individual may be asked to step down from their position of authority or influence in SOGS, and from which the improper benefit is derived.
            \item Bringing forward the matter to the Speaker to be resolved by Council or an ad hoc tribunal (as described in 2.2.3)
		\end{longenum}
        \item Council or an ad hoc tribunal may determine that a conflict does exist, and thus may recommend or impose the following (in adherence with 2.3, above), in order of desirability:
		\begin{longenum}[ label*=\arabic*., align=left]
			\item The conflicted individual be asked to cease participating in the situation generating the conflict, either by noted abstention during voting on the issue, by leaving the room during a committee meeting in which the issue arises, or some other similarly appropriate and generally benign measure of resolution. 
            \item The conflicted individual may be asked to step down from their position of authority or influence in SOGS, and from which the improper benefit is derived.
            \item The conflicted individual may be censured and face no further disciplinary actions.
            \item The conflicted individual may be censured and face additional disciplinary measures as outlined in 4.3.1 and 4.3 .2, or more severe measures such as: 
            \begin{longenum}[ label*=\arabic*., align=left]
				\item A ban from committee proceedings, or other specific activities, or
                \item A ban from the Society not exceeding twelve (12) months.
                   \index{Conflict of Interest!Punitive Measures}        

			\end{longenum}
		\end{longenum}
		\item In all instances where a judgement is being proffered, rather than a mere recommendation, the matter shall be presented to Council for formalization and documentation.
   \index{Conflict of Interest!Presenting Formal Findings to Council}        

        \begin{longenum}[ label*=\arabic*., align=left]
			\item Recommendations, provided they achieve a resolution, need not be formalized in any capacity as these should be interpreted as successful private mediation. Such resolutions should never be brought to the formal attention of Council or recorded in its minutes.
\index{Conflict of Interest!Presenting Informal Resolution to Council}        

		\end{longenum}
        \item In all instances where punitive measures are taken against a conflicted individual, said measures shall be rationalized in absolutely clear writing that the punishment does not exceed the charge against the conflicted person.
        \item Punitive measures should be avoided except under the most egregious circumstances.
           \index{Conflict of Interest!Punitive Measures}        

	\end{longenum}    




\end{longenum}