\section{Example Section}

This is an example preamble paragraph.  

 \begin{longenum}[ label*=\thesection.\arabic*., align=left]
 \item This is a first item
 \item All first level items need the ``label*=\textbackslash thesection.\textbackslash arabic*." attribute tag in the longenum command in order to have the section number in the numbering scheme.  
  \begin{longenum}[ label*=\arabic*., align=left]
  \item this is a second-level item
  \item All second-level items only need the label*= \textbackslash arabic* attribute, since it appends a new Arabic numeral to the previous level's numbering. 
  \end{longenum}
  \item this is a return to first-level items.  Notice that the numbering continues where it left off?
  \end{longenum}
  
\section{Example Index}
 \begin{longenum}[ label*=\thesection.\arabic*., align=left]
 \item This is how to create an index in \LaTeX
 \item The index is created from tags created from the \textbackslash label command. Example: \textbackslash index\{SOGS\}
 \item A second level index is created as follows:  \textbackslash index\{SOGS!Second Level\}
 \item A third level index is created as follows:  \textbackslash index\{SOGS!Second Level!Third Level\}
 \item Index works down to the third level
 \item When index levels are spelled identically, they will be grouped together in the index.  In other words, the Index is CaSe SenSiTive and White space Sensitive.
\item ''\{First Level!Second Level\}" not equal to ''\{First Level! Second Level\}"  
   \end{longenum}