
\section{Committees of the Society}
\index{Committee|(}
\subsection{Creating and Disbanding}
\index{Committee!Creation}
\index{Committee!Disbanding}
\index{Committee!Ad Hoc}
\begin{longenum}[ label*=\thesubsection.\arabic*., align=left]
	\item Any motion to strike a new committee (whether standing or ad hoc) shall be accompanied by an outline of the type, membership, goals and duties, Official Liaison, and, when appropriate, a title for the Chairperson of the committee. In the case of ad hoc committees, this outline need neither be in writing or made known in advance. 
    \item Proposals to create, modify, or disband standing committees are normal amendments to the Bylaws.
    \item Ad hoc committees shall be created, modified, or disbanded by Council.Each ad hoc committee, when its final report is accepted, shall be considered disbanded.
    \item Once a proposal for a new committee has been accepted by Council, the Official Liaison shall be responsible for arranging an organizational meeting. 
\end{longenum}

\subsection{Duties of the Chairperson}
\index{Committee!Chairperson}
\index{Chairperson!General Duties}
The Chairperson shall:

\begin{longenum}[ label*=\thesubsection.\arabic*., align=left]
	\item be chosen by the committee from amongst its members. If a committee has no chair, the Official Liaison to that committee shall become the interim chair until a chair is elected.
    \index{Committee!Chairperson!Selection}
    \index{Chairperson!General Duties!Selection}
    \item have the authority to appoint members to the committee, for a non-renewable term extending to the next Council meeting. 
    \index{Committee!Chairperson!Appointing new members}
    \index{Chairperson!General Duties!Appointing new members}
    \item submit to Council a list of members appointed to the committee and removed from the committee since the previous Council meeting.
    \item shall notify the Speaker immediately of any vacancies on the committee with the intent of advising the Speaker to advertise said vacancy in relevant Society media (e.g. Society newsletter, Council package, etc). Please refer to the Speaker's responsibilities, Bylaw 2.17.
    \index{Committee!Chairperson!Notifying Speaker of vacancy}
    \index{Chairperson!General Duties!Notifying Speaker of Vacancy}
    \item have meetings called and notices (oral or written) distributed
    \index{Committee!Chairperson!Agenda}
    \index{Chairperson!General Duties!Agenda}
    \item prepare a meeting agenda
    \item chair meetings
    \item have brief minutes of each meeting taken and prepared in writing
    \index{Committee!Chairperson!Minutes}
    \index{Chairperson!General Duties!Minutes}
    \index{Minutes!Committee}
    \begin{longenum}[ label*=\arabic*., align=left]
		\item The Chairperson shall submit committee minutes to the Speaker for inclusion into the Council package each month; committee chairpersons shall follow Robert Rules of Order guidelines on the composition of proper minutes (11th edition, section 48, pp. 468-476).
	\end{longenum}
    \item maintain a committee file to be kept in the Society's office
    \index{Committee!Chairperson!Committee Folder}
    \index{Chairperson!General Duties!Committee Folder}
    \item maintain a Committee Policy document and make it available in the Society's office.
    \index{Committee!Chairperson!Maintaining Policy Document}
    \index{Chairperson!General Duties!Maintaining Policy Document}
    \index{Policy Manual!Committee Level Document}
    \item present a written report at least once a year at the Annual General Meeting. Motions pertaining to the committee shall be submitted separately from the report, and shall be accompanied by a statement including names of proposer and seconder.
    \index{Committee!Chairperson!Annual Report}
    \index{Chairperson!General Duties!Annual Report}
    \item co-ordinate with Executive members when necessary. 
    \index{Committee!Chairperson!Coordinating with Executive}
    \index{Chairperson!General Duties!Coordinating with Executive}
    \item be responsible for ensuring that committee members execute their duties.
    \item assume other duties within the mandate of the committee in consultation with the Official Liaison.
\end{longenum}

\subsection {General}
\begin{longenum}[ label*=\thesubsection.\arabic*., align=left]
	\item Each committee shall obtain, as necessary, input on projects within its mandate from members of the Society and, where appropriate, from outside sources. 
    \item Each Committee shall have an Official Liaison, who shall be either an Executive member or the Speaker of the Society. 
    \index{Committee!Official Liaison}
    \item The Official Liaison shall be considered an ex-officio, non-voting member of the committee.
    \index{Committee!Official Liaison!Ex-Officio}
    \item In any committee of the Society with a composition not exceeding three (3) members, all voting members must be present in person for business to be transacted.
        \index{Committee!Quorum}
        \index{Quorum!Committee!Less than 3 Members}
    \item In any committee of the Society with a composition exceeding 3 members, a majority of one--half plus one must be present in person or by proxy for business to be transacted. 
        \index{Quorum!Committee!More than 3 Members}
    \item All business transacted in the absence of quorum is null and void. 
    \item For the purposes of determining quorum, membership of any committee of the Society shall constitute only those members who are currently present or who have attended at least one meeting of the same committee in the past.
     \index{Committee!Quorum!Determining Quorum}
     \index{Quorum!Committee!Determining Quorum}
    \item Members of a committee appointed by the Chairperson shall be permanent members of the committee after their names have been submitted to Council, as per Bylaw 9.2.3, unless a motion to object to a member appointed to a committee via Bylaw 9.2.2 is passed by Council.
    \item In the event of the failure of a voting member of a committee to attend two meetings during their term either in person or by proxy, the chair of the committee shall remind the member in writing of their duty to attend committee meetings. If, after notice has been given, a subsequent meeting be missed by the member, the chair may, at the discretion of the other members of the committee, remove the member from the committee.
    \index{Committee!Membership!Removal for Absence}
    \index{Absenteeism}
\end{longenum}

\subsection{Committee Policies}
\index{Committee!Policy Document|(}
\begin{longenum}[ label*=\thesubsection.\arabic*., align=left]
	\item Each committee shall have a Committee Policy document, distinct from other committees' policy documents or any other policy document. 
    \item A Committee Policy document describes any policies used to conduct business in the committee or membership in the committee. Notwithstanding guidelines or restrictions mentioned in Robert's Rules of Order, each committee has the power, via its Committee policy document, to:
    \index{Committee!Policy Document!Role}
    \begin{longenum}[ label*=\arabic*., align=left]
		\item Allow email voting, mail voting and/or fax voting and to specify the conditions under which these voting methods are permissible for conducting official business.
            \index{Committee!Policy Document!Email Voting}
        \item Limit membership to a specific number.
            \index{Committee!Policy Document!Membership Cap}
        \item Establish a Steering Committee, where the Steering Committee has its own policies, as per 9.2, distinct from those of the committee as a whole. Official business done by the Steering Committee shall be automatically ratified and taken to be official business done by the committee as a whole. Motions and reports from the Steering Committee shall be considered official business of the committee under Bylaw 15.3.7.
	\end{longenum}
   
    \item All amendments to a Committee Policy document must be approved by the committee as official committee business. Amendments to a Committee Policy document come into effect immediately. Notice of any amendment to the Committee Policy document must be given to the Official Liaison of the committee within two (2) business days of being approved.    \index{Committee!Policy Document!Amendment}
    
\end{longenum}
\index{Committee!Policy Document|)}
\subsection{Standing Committees}

    
\subsubsection{The Academic Committee}

\begin{longenum}[ label*=\thesubsubsection.\arabic*., align=left]
	\item shall have a Chair;
    \item shall support academic events of interest to graduate students
    \index{Committee!Academic!Mandate}
    \item shall be responsible for the Western Research Forum and other awards programs that the Society may wish to administer.
    \index{Committee!Academic!Western Research Forum}
    \index{Western Research Forum}
    \item shall advertise organized and supported activities.

\end{longenum}

\subsubsection{The Graduate Student Teaching Awards Committee}
\index{Committee!GSTA!Mandate}
\index{Chairperson!GSTA!Mandate}
\begin{longenum}[ label*=\thesubsubsection.\arabic*., align=left]
	\item shall have a Chair who shall be known as the Graduate Student Teaching Awards Coordinator;
    \item shall have at least one member from each of the divisions of the School of Graduate and Postdoctoral Studies;
    \index{Committee!GSTA!Composition}
    \item shall have one member appointed by PSAC 610;
    \index{Committee!GSTA!PSAC 610}
    \index{PSAC 610!Membership on GSTA Committee}
    \item shall administer the Graduate Student Teaching Awards in conjunction with the School of Graduate and Postdoctoral Studies and PSAC 610.
    \item shall award an equal number of Teaching Assistants teaching in each of the following areas: Arts, Biosciences, Physical Sciences, and Social Sciences.
    \index{Committee!GSTA!Equal Distribution}
\item shall provide all legitimate nominees for the GSTA with a certificate, upon request within one year of receiving feedback, that can be added to a teaching
dossier.
\item shall provide all nominees for the GSTA with a copy of comments (anonymized where necessary) pending
release by their nominees via automated e-mail.

\end{longenum}

\subsubsection{The Bylaws and Constitution Committee (BCC)}
\begin{longenum}[ label*=\thesubsubsection.\arabic*., align=left]
	\item shall devise new and review old Bylaws to reflect the current needs and wishes of the Society and to allow the specific elaboration, interpretation, and application of the Society's Constitution 
    \index{Committee!Bylaws and Constitution!Mandate}
    \item shall ensure that the Bylaws be kept updated
    \item shall react to motions of Council that may impact on the Bylaws and Constitution
    \item shall review the Bylaws and Constitution at least once per year
    \item shall have a chair who
    \begin{longenum}[ label*=\arabic*., align=left]
        \index{Chairperson!Bylaws and Constitution!Serve as Speaker Pro Tem}
		\item shall be the Speaker's first choice to serve as Speaker pro tem
        \item shall be familiar with Robert's Rules of Order
          \index{Chairperson!Bylaws and Constitution!Familiarity with Roberts Rules}
          \index{Deputy Speaker!Chairperson of BCC}
        \item as requested, shall advise Councillors on the wording of motions and on proper procedure
          \index{Chairperson!Bylaws and Constitution!Advise Members}
        \item shall assist the Speaker in the performance of the Speaker's duties.
          \index{Chairperson!Bylaws and Constitution!Assist Speaker}
        \item serves, ex--officio, as the Deputy Chief Returning Officer of the Society.
          \index{Chairperson!Bylaws and Constitution!Ex--Officio DCRO}
          \index{Deputy Chief Returning Officer!Chairperson of BCC}
     
        \end{longenum}
    \item shall, at all times, maintain a Deputy Chairperson.
    \index{Committee!Bylaws and Constitution!Deputy Speaker}
    \item shall have the chairperson of the Policy Committee as voting ex--officio member.
     \index{Committee!Policy!BCC Chairperson as ex Officio}
     \index{Chairperson!Bylaws and Constitution!Ex--officio member of Policy Committee}
\end{longenum}

\subsubsection{The Finance Committee (FC)}
\begin{longenum}[ label*=\thesubsubsection.\arabic*., align=left]
	\item shall have a Chair;
	\begin{longenum}[ label*=\arabic*., align=left]
		\item who shall also sit on the SOGS/PSAC 610 Joint Thesis Completion Fund Committee, or send a designate.
		\end{longenum}
    \item shall monitor spending and consider amendments to the budget for submission to Council for approval
    \index{Committee!Finance!Mandate}
    \index{Committee!Finance!Budget Amendments}
    \index{Budget!Amendment!Consulting Finance Committee}
    \item shall, whenever possible, assist in collecting and organizing data on overall student support
    \item shall assist the Vice-President Finance to draft a budget, subject to approval by Council, while considering information such as previous budgets, previous actual spending, and the stated priorities of Council
        \index{Committee!Finance!Assist Vice-President Finance}
        \index{Budget!Finance Committee!Assist Creation}
    \item shall recommend the fee(s) to be levied on full and associate members of the Society to support the operation of the Society at the Annual General Meeting
    \item shall monitor the long-term investments of the Society and make recommendations on their management to the Vice-President Finance.
   \item Shall, with the Vice-President Finance, be responsible for overseeing the Societys' emergency loan program and similar or equivalent programs for graduate students.
        \index{Committee!Finance!Monitor Long-Term Investments}
\end{longenum}

\subsubsection{The Bursary and Subsidy Committee (BSC)}
\begin{longenum}[ label*=\thesubsubsection.\arabic*., align=left]
	\item shall have a Chair;
	 \begin{longenum}[ label*=\arabic*., align=left]
		\item who shall also chair the SOGS/PSAC 610 Joint Thesis Completion Fund Committee, or send a designate.
		\end{longenum}
    \item shall, whenever possible, assist in collecting and organizing data on overall student support; 
        \index{Committee!Bursary and Subsidy Committee!Mandate}
    \item shall assist the Vice-President Finance in administering the Society, Grad Club and external bursaries and subsidies;
    \item shall make reasonable attempts to have representatives of at least one member from each Category of students (Cat I and Cat II);
     \index{Committee!Bursary and Subsidy Committee!Representation Quota}
    \item shall make reasonable attempts to have representatives of at least one student in each of the following divisions: Arts/Humanities, Sciences, and Social Sciences.
\end{longenum}

\subsubsection{The Graduate Club Committee (GCC)}
\begin{longenum}[ label*=\thesubsubsection.\arabic*., align=left]
 \index{Committee!Grad Club Committee!Mandate}
	\item the Grad Club Committee shall monitor spending and consider amendments to the Grad Club budget for submission to Council for approval, excluding necessary operation costs.
    \item shall review financial statements and revised budgets of the Grad Club
     \index{Committee!Grad Club Committee!Grad Club Oversight}
    \item shall review prices of products and services in the Grad Club
    \item shall recommend new policies to Council
    \item shall review old policies and forward revisions for approval by Council
     \index{Committee!Grad Club Committee!Review Grad Club Policies}
    \item shall approve Grad Club entertainment policy
    \item shall review suggestions received shall recommend honourary and associate Grad Club memberships to Council
    \item shall assist the Grad Club Manager to draft a budget, subject to approval by Council, while considering information such as previous budgets, previous actual spending, and the stated priorities of Council.
     \index{Committee!Grad Club Committee!Assist Budget Making}
      \index{Budget!Grad Club!Grad Club Committee Consultation}
\end{longenum}

\subsubsection{The International Graduate Students' Issues Committee (IGSIC)}
\begin{longenum}[ label*=\thesubsubsection.\arabic*., align=left]
 \index{Committee!IGSIC!Mandate}
    \item shall have a Chair;
    \item shall assess the particular needs of international students and disseminate relevant information through the Society International Graduate Student Listserv and
the Society International Graduate Student Facebook Group;
\item shall oversee and update, in consultation with the Society Communications
Administrator and the Vice-President Student Services, the Society International Graduate Student
Issues Committee (IGSIC) Facebook group;
    \item shall advocate on behalf of such students with the administration, specifically Western International, the IESC, and the School of Graduate and Post-Doctoral Studies, as necessary.
\end{longenum}

\subsubsection{The Orientation and Social Committee (OSC)}
\begin{longenum}[ label*=\thesubsubsection.\arabic*., align=left]
	\item shall have a Chair;
   \index{Committee!Orientation and Social Committee!Mandate} 
    \index{Committee!Orientation and Social Committee!Handbook/dayplanner}
    \item  shall organize and host the Society's Orientation (third week of September) and social events throughout the year for all Society members
     \index{Committee!Orientation and Social Committee!Orientation}
    \item  shall endeavour to coordinate with representations from each Graduate 
    Association/Department for Orientation and social events throughout the year;
     \index{Committee!Orientation and Social Committee!Social Events}
    \item  shall endeavour to offer, organize and run athletic and/or non-alcohol focused events in an effort to better reflect the ever changing needs/interests/demographics of graduate students; 
     \index{Committee!Orientation and Social Committee!Sports Teams}
    \item shall work in conjunction with the Grad Club manager and the Grad Club Committee to host events at the Grad Club whenever possible.
    \item shall oversee, in consultation with the Vice-President Student Services, the Orientation and Social Committtee Social
Sponsorship Fund.

\end{longenum}
 
\subsubsection{GradCast  Editorial Board}
\begin{longenum}[ label*=\thesubsubsection.\arabic*., align=left]
	\item  shall have a Chairperson who shall be known as the Managing Editor;
    \item shall aim to have at least one member from each of the four divisions: Arts, Biosciences, Physical Sciences and Social Sciences;
     \index{Committee!Gradcast!Representation Quota}
    \item shall advertise for submissions;
    \item shall maintain instructions for contributors.
\end{longenum}
 
\subsubsection{The Graduate Student Issues Committee}
\begin{longenum}[ label*=\thesubsubsection.\arabic*., align=left]
	\item shall have a Chair;
         \index{Committee!GSIC!Mandate}
    \item shall monitor, assess, and respond to issues pertaining to the quality and accessibility of graduate education;
    \item shall work to keep graduate students informed of these issues;
    \item shall select non--executive members of the Society to be dispatched as delegates to general meetings of the Canadian Federation of Students.
         \index{Committee!GSIC!Select Non--Executive Delegates to CFS}
\end{longenum}

\subsubsection{The Policy Committee}

\begin{longenum}[ label*=\thesubsubsection.\arabic*., align=left]
	\item shall have a Chair who:
    \begin{longenum}[ label*=\arabic*., align=left]
		\item shall maintain the motions database;
        \item shall maintain the policy database;
        \index{Chairperson!Policy!Motions Database}
        \index{Chairperson!Policy!Policy Manual}
        \index{Committee!Policy!Motions Database}
        \index{Committee!Policy!Policy Manual}
        \item shall sit as voting, ex--officio member of the BCC.
        \index{Chairperson!Policy!Ex--Officio BCC Member}
	\end{longenum}
    \item shall have one voting ex--officio member: the Deputy 
Speaker;
  
    \index{Committee!Policy!Mandate}
	\item shall review the motions database in order to discern and/or devise policy from it for systematic inclusion in a distinct Policy Manual which will be publicly available in the same manner as other documents of the Society;
    \index{Policy Manual}
    \index{Motions Database}
    \item shall develop a system for cross--referencing the motions database from within the Policy Manual;
    \item shall draft motion--based policies so that the original intent of the motion is not substantially altered;
    \item shall react to motions of Council that affect the Policy Manual;
    \item shall meet at least once between meetings of Council.
    
\end{longenum}

\subsubsection{Equity Issues Committee} 
\begin{longenum}[ label*=\thesubsubsection.\arabic*., align=left]
    \index{Committee!Equity Issues Committee!Mandate}
	\item shall discuss and make recommendations to Council on equity issues affecting graduate students.
    \item shall serve as a liaison between the Society and other groups and programs concerned with equity issues on campus and in the community.
    \item shall work to increase awareness of equity issues on campus, particularly within the graduate student community, through events and publicity campaigns.
    \item shall have all Commissioners sit as non--voting, ex--officio members of the committee.
    \index{Commissioner!Ex-Officio on of Equity Committee}
    \index{Committee!Equity Issues Committee!Commissioner as ex--Officio}
    
\end{longenum}

\subsubsection{Health Plan Committee (HPC)} 
\begin{longenum}[ label*=\thesubsubsection.\arabic*., align=left]
	\item shall have a Chair;
    \index{Health Plan!Committee}
    \item shall review and recommend changes to the society's health plan and health plan policies; 
    \index{Committee!Health Plan!Mandate}
    \item shall review health plan contracts and make recommendations to council ;
    \index{Health Plan!Review}
    \item shall include the accounts manager as a non--voting member;
    \index{Committee!Health Plan!Ex-Officio}
    \item shall work in consultation with the Vice--President Student Services to help run and promote mental health and wellness initiatives for graduate students. 
    \index{Committee!Health Plan!Mental health and wellness}
    
\end{longenum}

\subsubsection{Sustainability Committee (SC)}
\begin{longenum}[ label*=\thesubsubsection.\arabic*., align=left]
	\index{Committee!Sustainability!Mandate}
    \item shall have a Chair who:
    \begin{longenum}[ label*=\arabic*., align=left]
		\item shall represent the Society on the University Sustainability Committee;
        \item shall report to Council once a term.
	\end{longenum}
    \item shall assess the sustainability practices of the Society, including activities of the executive, the Society's office, and council;
    \item shall assess the sustainability practices of the Grad Club, and work to make viable recommendations for improvement, so that it  may be a progressive leader for sustainable development  on the campus. 
    \item shall co-ordinate campus advocacy for sustainable policies and practices on behalf of the Society.
\end{longenum}

\subsubsection{Graduate Peer Support Committee (GPS}
\begin{longenum}[ label*=\thesubsubsection.\arabic*., align=left]
	\index{Committee!Graduate Peer Support Committee!Mandate}
	\item GPS committee will have a chair who shall be known as the coordinator of the committee.
	\item  GPS is focused on graduate student wellness through providing community space, events and funding opportunities.
 \begin{longenum}[ label*=\arabic*., align=left]
\index{Committee!Graduate Peer Support Committee!Mandate}
\item shall coordinate the Graduate Peer Support program,
\item shall coordinate the Wellness Joint fund
\item shall coordinate the Society's Food Bank
\item shall liaise with Wellness Ambassadors
\end{longenum}

\end{longenum}
\subsection{Commissions}
\index{Commission}
\subsubsection{General}
\begin{longenum}[ label*=\thesubsubsection.\arabic*., align=left]
	\item Commissions are committees with the following exceptions:
    \begin{longenum}[ label*=\arabic*., align=left]
		\item The chairperson is the Official Liaison to Council, unless otherwise specified in this (9.6) section.
	\end{longenum}
\end{longenum}
\subsubsection {Women's Concern Commission}
\index{Women's Consern Commission!Mandate}
\index{Commission!Women's Consern Commission!Mandate}
\begin{longenum}[ label*=\thesubsubsection.\arabic*., align=left]
	\item shall discuss and make recommendations to Council on issues affecting graduate students who are constituency members.
    \item shall serve as a liaison between the Society and other groups and programs concerned with issues of concern to women on campus and in the community.
    \item shall work to increase awareness of issues affecting women and transgendered people on campus, particularly within the graduate student community, through events and publicity campaigns.
    \index{Women's Concern Commission!Membership}
\index{Commission!Women's Concern Commission!Membership}
    \item members of the Commission must be graduate students who are women and/or individuals who identify as transgender.

\end{longenum}

\newpage
\subsection{Official Liaison Table}
\begin{center}
    \begin{tabular}{  c | c }
\index{Committee!Official Liaison Table}
\index{Executive, The!Official Liaison Table}
    Committee & Official Liaison  \\ \hline 
    Academic & VP Academic \\ 
    BCC & Speaker \\ 
    Equity Issues & VP Advocacy \\ 
    Finance & VP Finance  \\ 
    Grad Club & VP Finance  \\ 
    Graduate Student Issues & VP Advocacy \\ 
    Graduate Student Teaching Awards & VP Academic \\ 
    Health Plan & VP Student Services \\ 
    International Student Issues &  VP Student Services \\ 
    Orientation and Social & VP Student Services \\  
    Policy & President \\ 
    Sustainability &VP Advocacy \\ 
    Gradcast & VP Academic \\ 

\end{tabular}
\end{center}

\newpage
\index{Committee|)}

\section{Personnel}

\begin{longenum}[ label*=\thesection.\arabic*., align=left]
	\item All employment positions created by the Society must be approved by Council.
    \index{Personnel!New Positions}
    \index{Council!New Society Personnel}
    \item Each employment position created by the Society must have an explicit job description prepared by the Executive or by an appropriate officer of the Society and shall be approved by Council. 
    \item Council shall establish a Personnel manual pertaining to all aspects of employment by the Society and shall be responsible for approving changes to the manual.
    \item The President shall coordinate the contract renewal process for the employees at the Society. Contract negotiations shall be conducted in consultation with the Executive.
    \index{President, The!Personnel Contract}
    \item The President and VP Finance, with at least one additional executive member, shall strike a subcommittee of at least three executive members to convene and conduct performance reviews of all Society salaried employees, in addition to the Grad Club Manager, excluding the Society's Grad Club employees. This subcommittee shall conduct this review annually by the end of March and prior to a contract renewal discussion.
    \index{President, The!Contract Review Subcommittee}
    \index{Vice-President Finance!Contract Review Subcommittee}
    \index{Grad Club!Contract Review Subcommittee}
\end{longenum}

\newpage

\section{Conflict of Interest}
\index{Conflict of Interest|(}
\begin{longenum}[ label*=\thesection.\arabic*., align=left]
	\item A conflict of interest arises when any current member of Council, the Society's Executive or Non-Executive officers, committee member, or other elected or appointed position of the Society has or could be seen to have the opportunity to use the authority, knowledge or influence derived from one's position in order to provide personal or financial gain to the individual in question, or a member of the individual's family, or else an agency with which the individual is employed. The following instances could be  considered conflicts of interest, though conflicts are not limited to these instances: 
    \index{Conflict of Interest!Definition}
    %\begin{longenum}[ label*=\arabic*., align=left]
    \begin{itemize}
    	\item One is involved in both the solicitation of, and rendering of services or products to the Society, and for which discretionary authority may be exercised at any stage  of the commissioning of said services or products.
        \item One is involved in either the solicitation or rendering of services or products to the Society wherein one stands to make personal financial gains, for instance as a shareholder, contract employee, or investor.
        \item One is participating in the hiring, contract review, or investigation of an immediate relative, spouse, in-law, step-child, or third-party business partner for the Society. 
        \item One places or has placed the needs or demands of a third party or agency above the stated or acknowledged needs of the Society. 
        \item One is involved as an adjudicator in any of the Society's adjudication processes wherein one is also an appellant, complainant or defendant, or in which one has provided material evidence.
        \item An Executive or Non-Executive Officer (excluding
Commissioners, Deputy Speaker, CRO, and DCRO) participates outside of an administrative role in a draw, lottery or similar event organized by the Society or its affiliated businesses (ex. the Grad Club). The same applies to TA Awards, Western Research Forum, all bursary or grant programmes, and similar matters run by the Society. Other instances of Conflict of Interest not listed above, but which are consistent with the spirit of the law, may be considered on an ad hoc basis provided they meet the minimum standard of demonstrable personal or financial benefit to the individual or a member of the individual's family,  or else an agency with which the individual is employed.
    \end{itemize}
    %\end{longenum}
    \begin{longenum}[ label*=\arabic*., align=left]
	\item Matters of common interest, referring to instances in which any member of the Society may have equal opportunity to  benefit from an arrangement of any type, are as a general  rule not considered a Conflict of Interest. That is to say, the interest must be specific to an identifiable individual or exclusive small group (business partnership, family, and the like). 
    \index{Conflict of Interest!Common Interest}
	\end{longenum}
    \item In all instances above, a member of the Society who perceives oneself to be in a Conflict of Interest, may do the following to extricate themselves from the conflict:
    \begin{longenum}[ label*=\arabic*., align=left]
		\item Declare one's conflict, and
        \begin{longenum}[ label*=\arabic*., align=left]
			\item Recuse oneself from all participation, including administrative, in the activity in which one deems oneself to be in conflict, or
            \item If the conflict of interest is such that one can no longer discharge their duties to the Society, one may resign from one's position of authority or influence, provided doing so resolves the conflict in question.
		\end{longenum}
        \item Pursuant to adopting any of the actions therein but only after the fact does not resolve the conflict, and may still result in sanctions or censure (see Bylaw 11.3 below).
	\end{longenum}
    \item The procedure for resolving any perceived and either undeclared or unresolved Conflict of Interest is governed by the Conflict of Interest Resolution Policy Document, located in the Society's office and on the Society's website.
    
\end{longenum}
\index{Conflict of Interest|)}
\newpage

\section{Finances}
\index{Finances}
\begin{longenum}[ label*=\thesection.\arabic*., align=left]
	\item An individual executive member can only authorize the spending of \$100 via a  UWO or U.S.C. account number for budgeted line items. Amounts in excess of \$100 are subject to normal signing authority, immediately below.
    \item Any withdrawal of funds from the Society's or Grad Club's bank accounts requires two signatures, one of which must be the President or Vice  President Finance.
    \index{Finance!Signing Authority}
    \index{Vice--President!Signing Authority}
    \index{President, The!Signing Authority}
    \index{Executive, The!Signing Authority}
    \begin{longenum}[ label*=\arabic*., align=left]
    	\item In the case of a withdrawal from the Society's accounts, the second signature may be any member of the executive that did not provide the first signature, or the Society's Accounts Manager. 
        \item In the case of a withdrawal from the Grad Club's accounts, the second signature may be any member of the executive that did not provide the first signature, or the Grad Club Manager.
        \index{Grad Club!Signing Authority}
    \end{longenum}
    \item Notwithstanding the above, the Grad Club Goods account requires only one signature, which must be the President's, Vice President Finance's, or the Grad Club Manager's. Similarly, use of the Grad Club Credit Card shall require only one signature, which must be the Grad Club Manager's. 
    \item Two executive members, one of whom must be the President or Vice-President Finance, are required to sign any loan agreements.
    \index{Finance!Loan Authority}
    \index{President, The!Loan Authority}
    \index{Vice-President Finance!Loan Authority}
    \item The President and one other member of the Executive shall sign all agreements on behalf of the Society, unless otherwise determined by Council. The President and the Vice-President Finance shall sign all agreements relating to the long-term investments of the Society and to all agreements requiring the withdrawal of funds from the Society's bank accounts, unless otherwise determined by Council.
    \item Proposals for amendments to the budget shall be received by the Vice-President Finance and shall be referred to the Finance Committee. The Finance Committee shall present the proposals with the Finance Committee's recommendations to Council within eight weeks of the Vice-President Finance's receipt of the proposals.
    \index{Finance!Budget}
    \index{Budget!Preparation!Finance Committee}
    \index{Budget!Preparation!Vice-President Finance}
    \index{Committee!Finance!Budget}
    \index{Vice-President Finance!Budget}
    
    \item The Society's budget shall be approved at a General Meeting
    \index{General Meeting!Budget Approval}
    \item Fees
    \begin{longenum}[ label*=\arabic*., align=left]
		\item The administrative and membership fees to be levied on full and associate members of the society shall be adjusted annually to a maximum of the Ontario Consumer Price Index
        \index{Budget!Fees!Consumer Price Index}
        
        \item All dues, fees and levies shall be deliberated by the finance committee
during the annual budget preparation and approved at the Annual General Meeting of the
Society.

		\item Any increase over and above the maximum increases listed above shall be introduced as a separate levy at the Annual General Meeting of the Society.
        \index{Budget!Fee Changes}
        \index{General Meeting!Membership Fees}
        
        
	\end{longenum}
    \item The Society will issue grant cheques to departmental graduate student organizations only upon receipt of confirmation that these organizations are duly constituted entities. A duly constituted entity shall be defined as an organization with a bank account in the name of the organization. Cheques will be made out to these organizations, not to individuals. Organizations are responsible for confirming their status with the Society. For those departments that do not fulfil this criteria, the Society will hold the funds in trust payable upon application to the Society by a duly elected Councillor.
    \index{Departmental Grants}
    \index{Finance!Departmental Grants}
    \item An independent external agent shall prepare a statement of the Society's financial position for the preceding year. This report shall be presented to Council in the fall term.
    \index{Finance!Annual External Review}
\end{longenum}
\newpage



\section{Graduate Support Programs}
\index{Graduate Support Programs}
\begin{longenum}[ label*=\thesection.\arabic*., align=left]
	\item Emergency Loans, travel grants, and child care grants administered to Society members shall be the joint responsibility of the President and Vice-President Finance.
\index{Graduate Support Programs!Emergency Loans}
\index{Graduate Support Programs!Travel Grants}
\index{Graduate Support Programs!Child Care}
\index{President, The!Graduate Support Program Administration!Emergency Loans}
\index{President, The!Graduate Support Program Administration!Travel Grants}
\index{President, The!Graduate Support Program Administration!Child Care}
\index{Vice-President Finance!Graduate Support Program Administration!Emergency Loans}
\index{Vice-President Finance!Graduate Support Program Administration!Travel Grants}
\index{Vice-President Finance!Graduate Support Program Administration!Child Care}
    \item The total amount of money available for these programs and the maximum amount an individual receives shall be determined annually at budget time by recommendation of the Vice-President Finance for approval by Council.
    \index{Graduate Support Program!Administration!Maximums}
    \index{Budget!Graduate Support Program!Maximum Help}
    \index{Vice-President Finance!Graduate Support Program Administration!Maximum Funding}
    \item An administrative charge may be levied on overdue loans.
    \item The names of students receiving emergency loans and child care grants are confidential.
    \index{Graduate Support Programs!Confidentiality}
    \item The names of persons receiving travel grants will be published in the  Society's newsletter.
\end{longenum}
\newpage

\section{Disclosure of Information}
\index{Disclosure of Information}
The Society of Graduate Students espouses an open information policy. In accordance with this policy:
\begin{longenum}[ label*=\thesection.\arabic*., align=left]
	\item All minutes of Council or the Executive, and, where compiled, of committees shall be available to all interested parties, with the exceptions noted below.
    \index{Minutes!Council}
    \index{Minutes!General Meeting}
    \index{Minutes!Executive}
    \index{Minutes!Committee}
    \index{General Meeting!Minutes}
    \index{Council!Minutes}
    \index{Executive, The!Minutes}
    \index{Committee!Minutes}
    \item Upon request of a full or associate member of the Society, the  President must confidentially reveal all details of employee salaries. This may be done  in person  or at a meeting of Council or the Executive.
    \index{Confidential Information}
    \index{Employee!Salary Information}
    \item For the purposes of these Bylaws, confidential minutes means non-public records taken during a meeting of Council, the Executive, or any committee of the Society.
    \item Confidential minutes may be viewed only by full and associate members of the Society.
    \index{Minutes!Confidential Minutes}
    \item Confidential minutes must be viewed in the presence of an Executive member or the Speaker or the Administrative Assistant. No notes may be taken or copies made of these minutes. In addition, the contents of confidential minutes must not be discussed with any person who is not eligible to view the confidential minutes. Confidential minutes must only be discussed in non public environments.
    \item Notwithstanding the above, all details of personnel matters are subject to the Society's Personnel Manual guidelines.
    \index{Minutes!Confidential Minutes!Exceptions}
    \item Employees of the Society are entitled, in the presence of an Executive member, to examine their own personnel files.
    \index{Employee!View Employee File}
    \item Matters pertaining to the sensitive academic and administrative problems of individuals shall be considered the private property of the individuals.
    \end{longenum}
\newpage

\section{Meetings}
\subsection{Reports}
\begin{longenum}[ label*=\thesubsection.\arabic*., align=left]
\index{Reports!Presenting Protocol!Council}
\index{Reports!Presenting Protocol!General meeting}
\index{Reports!Presenting Protocol!Roberts Rules of Order}
\index{Roberts Rules of Order!Presenting Reports}
\index{Reports!Oral}
\index{Reports!Written}
	\item Any written report or oral report presented at a General Meeting or Council Meeting must abide, as if it were debate, by the following sections of Decorum in Debate given in Robert's Rules of Order: Refraining from attacking a member's motives; Avoiding the use of members' names. 
    \begin{longenum}[ label*=\arabic*., align=left]
		\item For the purposes of decorum in reports, a member is taken to be any Member of the Society.
	\end{longenum}
\end{longenum}
\subsection{General Meetings}
\begin{longenum}[ label*=\thesubsection.\arabic*., align=left] 
	\item General meetings may be called by Council or by a signed petition of one hundred members of the Society.
    \index{General Meeting!Calling!Motion of Council}
    \index{General Meeting!Calling!Petition}
    \index{Petition!Call General Meeting}
    

    \item All members of the Society shall be entitled to attend General meetings. Only full and associate members  shall be entitled to:
    \index{General Meeting!Membership}
    \begin{longenum}[ label*=\arabic*., align=left]
		\item move, second, and vote on motions
            \index{Membership!General Meeting!Move Motions}
            \index{Membership!General Meeting!Second Motions}
            \index{Membership!General Meeting!Vote on Motions}
	\end{longenum}
    \item Quorum in the case of General meetings shall be equal to the number of Councillors constituting quorum for a Council meeting. Only full and associate members shall be counted in determining if quorum be present.
    \index{General Meeting!Quorum}
    \index{Quorum!General Meeting}
    \item Notice of a General meeting, including an agenda showing business to be transacted, shall be advertised in university publications and on campus bulletin boards at least one week in advance of the meeting. Business which has not been included in the advertised agenda requires a two-thirds majority in order for this business item to be added to the agenda. Motions arising from new business require a two-thirds majority to pass.
    \index{General Meeting!Notice of Meeting}
    \index{General Meeting!Agenda}
    \index{General Meeting!New Business!2/3 majority}
    \index{2/3 Majority!General Meeting!New Business}
    \index{2/3 Majority!General Meeting!Items Not on the Agenda}
    \index{Agenda!General Meeting}
	\item  Business for a General Meeting must be submitted 7 business days prior to the meeting by a Society Member 
    \index{Agenda!General Meeting!Fixed 7 Days Prior}
    \index{General Meeting!Agenda!Fixed 7 Days Prior}
    \begin{longenum}[ label*=\arabic*., align=left]
		\item electronically, from a UWO e-mail address, or;
        \item by a written, signed submission to the Society Office.
        \index{Agenda!General Meeting!Submitting Business}
        \index{General Meeting!Agenda!Submitting Business}
	\end{longenum}
    \item The Society is not bound to circulate any material in a meeting agenda deemed by the Speaker to be:
    \begin{longenum}[ label*=\arabic*., align=left]
    \index{Agenda!False Statements}
    \index{Speaker, The!Agenda!False Information}
    \index{Agenda!Sensitive Information}
    \index{Speaker, The!Agenda!Sensitive Information}
		\item false; or
        \item pertaining to sensitive legal or fiscal negotiations currently in progress.
	\end{longenum}
    \item Minutes will be taken and will be available in the Society office after ratification at the subsequent Council meeting.
    \index{Minutes!Availability}
    \index{Minutes!Ratification}

    \item Every meeting shall be chaired by the Speaker, or in their absence, by a Speaker  \textit{pro-tem} elected by the members present.
    \index{General Meeting!Chair!Speaker}
    \index{General Meeting!Chair!Speaker \textit{pro-tem}}
    \item Decisions of a General meeting which are not contrary to the Society's Constitution and Bylaws shall be binding on Council.
    \index{General Meeting!Decisions Binding on Council}
    \index{Council!Decisions of General Meeting Binding on Council}
    \item There shall be an Annual General Meeting of the Society, to be held every February. This shall be a General  meeting' as defined above. This meeting shall not be held the same night as a Council meeting. 
    \index{General Meeting!Annual!Schedule}
\end{longenum}
\subsection{Council Meetings}
\index{Council}
\begin{longenum}[ label*=\thesubsection.\arabic*., align=left]
	\item The Executive shall call a meeting of Council at least monthly, with the exception of the month of December (in which a Council meeting shall not normally beheld unless called by the President).  Meetings of Council may also be called by the President, by a majority of Councillors, or by a signed petition of one hundred (100) members of the Society.
    \index{Council!Calling}
    \index{Executive, The!Calling Council Meeting}
    \index{President, The!Calling Council Meeting}
    \index{Petition!Calling Council Meeting}
    \item Written notice of the time and place of a meeting of Council shall be circulated to Council members at least one week prior to the meeting. This circulation shall be done via a package, called the Council mail-out, which shall also contain the agenda for the meeting, reports from every Executive member, Speaker's Rulings, a list of positions for election, and motions to be debated at the meeting.
    \index{Council!Mail-out!Agenda}
    \index{Council!Mail-out!Reports}
    \index{Reports!Council}
    \index{Executive, The!Reports!Monthly}
    \index{Elections!Non-Presidential!Council}
    \index{Council!Mail out!Non-Presidential Elections}
    \index{Agenda!Council!Motions}
    \index{Council!Mail out!Motions}
    \index{Agenda!Council!List of Elections}
    \index{Council!Mail out!Agenda}
    \item All Councillors and Executive members are expected to attend the Council meetings. 
    \index{Council!Attendance}
    \item All non-executive Officers, ex-officio members of Council and any other person may attend and may speak at Council if recognized by the chair, but may not move, second, or vote. Non-members of the Society may be excluded at Council's discretion. 
    \index{Council!Non-Voting Members}
    \index{Non-Executive!Council Voting}
    \item Business for a Council Meeting must be submitted seven (7) business days prior to the meeting by a Department Councillor or Executive of the Society.
	\begin{longenum}[ label*=\arabic*., align=left]
		\item electronically, from a UWO e-mail address, or;
        \item by a written, signed submission to the Society Office prior to the submission deadline.
        \index{Council!Agenda!Submission}
        \index{Agenda!Council!Submission}
     \end{longenum}
    \item Members of the Society wishing to submit business for consideration at council must do so through their department representative or the Society Executive.
    
	\item Committees of the Society wishing to submit business for consideration at Council must approve the business in question as a committee and must have it submitted by the chair of the committee.
    \index{Committee!Submitting Business to Council}
	\begin{longenum}[ label*=\arabic*., align=left]
		\item Committees of the Society wishing to submit a report to Council must approve the report as a committee, clearly marked as a majority report of the committee', and must have it submitted by the chair of the committee.
        \index{Committee!Majority Report}
        \index{Committee!Minority Report}
        \index{Council!Motions from Committees!Majority Report}
        \index{Council!Motions from Committees!Minority Report}

        \item Members of a committee of the Society wishing to submit a report to Council dissenting the opinion or business of that committee must submit the report, clearly marked as a minority report of the committee, endorsed by at least one named member of the committee. 
	\end{longenum}
    \item The Society is not bound to circulate any material in a Council Mailout deemed by the Speaker to be:
    \begin{longenum}[ label*=\arabic*., align=left]
		\item false; or
        \item pertaining to sensitive legal or fiscal negotiations currently in progress.
    \index{Agenda!False Statements}
    \index{Speaker, The!Agenda!False Information}
    \index{Agenda!Sensitive Information}
    \index{Speaker, The!Agenda!Sensitive Information}
	\end{longenum}
    \item Every meeting shall be chaired by the Speaker, or in their absence the Deputy Speaker, or in their absence, by a Speaker \textit{pro-tem} elected by Council.
    \index{Council!Chair!Speaker}
    \index{Council!Chair!Speaker \textit{pro-tem}}
    \index{Council!Chair!Deputy Speaker}
    \index{Deputy Speaker!Chair}
    \item A quorum in the case of Council shall constitute twenty (20) percent of the number of filled council seats, or twenty-four (24), whichever is greater. During the Summer Term (May through August) the quorum shall be set at fifteen (15) percent of filled council seats or 18, whichever is greater. All business transacted in the absence of quorum is null and void.
    \index{Council!Quorum!Summer}
    \index{Quorum!Council!Summer}
    \index{Council!Quorum!Fall/Winter}
    \index{Quorum!Council!Fall/Winter}
    \item Non-voting ex-officio members of Council shall not be counted for the purposes of determining quorum
    \index{Council!Quorum!Non-Voting Members}
    \index{Quorum!Council!Non-Voting Members}
    \index{Non-Voting Member!Council}
    \item Alternate Councillors
    \begin{longenum}[ label*=\arabic*., align=left]
		\item An alternate Councillor is a member of the Society that has been appointed by their department/ unit/constituency to serve as a
		\begin{itemize}
		\item Filled-seat alternate Councillor, who replaces a Councillor from their department/unit/constituency that intends to miss one or more consecutive meetings of Council, or
		\item Vacant-seat alternate Councillor, who serves as an interim representative for their department/unit/constituency in the event that there is a vacant Council seat.
		\end{itemize}	 
		An alternate Councillor will assume all the responsibilities of the regular Councillor for the duration of the regular Councillor's absence.
        \item A filled-seat alternate Councillor will be recognized by Council only if they present a completed copy of the official Society "Filled-Seat Alternate Councillor Form" to the Speaker prior to the Council meeting at which their temporary tenure is to begin.
        \item A vacant-seat alternate Councillor will be recognized by Council only if they present a completed copy of the official Society "Vacant-Seat Councillor Alternate Form" to the Speaker prior to the Council meeting for which their temporary tenure is to apply. 
        \index{Council!Alternate Councillors}
        \item An alternate Councillor shall count for quorum, and toward attendance for the purpose of their departmental grant. 
    \index{Council!Alternate Councillors!Quorum}
    \index{Quorum!Council!Alternate Councillors}
	\end{longenum}
    \item Business arising at a meeting of Council which is not placed in the Council mailout requires a two-thirds affirmative vote to pass.
    \index{2/3 Majority!Council!Items Not ON the Agenda}
    \index{2/3 Majority!Council!New Business}

    \item All original main motions and amendments must be in writing.
    \index{Council!Motions!Original Motion in Writing}
    \item Minutes shall be taken and shall be distributed to all Councillors before the next Council meeting.
    \index{Minutes!Council}
    \item Each department shall receive a grant from the Society every term, based on its size in terms of students and the number of council meetings at which it was represented.
    \item The number of students in a department for the purposes of determining councillor representation is the number of Associate Members divided by 3 (rounded up to the nearest whole number) added to the number of Full Members. 
        \index{Departmental Grants!Determining Number of Students}

    \item The degree to which a department is said to be "represented'' at a council meeting is based on the number of councillors recognized as having been in attendance within the ratified Society Council minutes of that meeting, divided by the total number of councillor positions that department has been allocated.
    		\item A councillor will be recognized as having been in attendance at a Council meeting for the purposes of the departmental grant if only the Councillor was present for one and a half hours, or two thirds of said meeting, whichever period of time is shorter.
    		\item Departments have up to two months of Council counted as full attendance regardless of actual attendance so long as their department has sent at least one Councillor to one full meeting over the course of the year.
    \index{Departmental Grants!Representation}
    \index{Council!Seat Apportionment}
    \item The formula by which the specific amount of a departmental grant will be determined as follows:
	\begin{longenum}[ label*=\arabic*., align=left]
		\item The number of students in a department will be multiplied by \$3.00. 		
        \item The result of the calculation in 2.15.3.19.1 will then be multiplied by that department's total percentage of representation at the term's Council meetings. This will be the dollar amount of the rebate earned that month.     
        \item Summing over the rebate earned for each meeting in a given term yields the precise dollar value of the departmental grant cheque that is issued by the Society to that department that term.
         \item Departmental grant calculations be based on the highest attendance count values over the course of the year. 
        \index{Departmental Grant!Cheque}
        
    \end{longenum} 
    
    \item A minimum amount of \$20.00 will be issued to departments that have fewer than 10 graduate students and a 100\% attendance rate at council meetings.
    \item This policy on departmental grant cheques is only to be applied to the eleven (11) regularly scheduled Society Council meetings held during one calendar year.
\end{longenum}
\subsection{Executive Meeting}
\begin{longenum}[ label*=\thesubsection.\arabic*., align=left]
	\item Executive members shall meet at least once a month and may request the attendance of any other person in a non-voting capacity. 
    \index{Executive, The!Executive Meetings!Schedule}
    \index{Executive, The!Executive Meetings!Non-Voting Invitee}

    \item Notice of the time and place of an Executive meeting must be circulated to all Executive members by the President.
    \index{Executive, The!Executive Meetings!Notice}

    \item A quorum shall be greater than fifty per cent (50\%)of the Executive positions.
    \index{Quorum!Executive Meeting}
    \index{Executive, The!Executive Meetings!Quorum}

    \item Written minutes must be recorded by an office staff member. The minutes shall be presented at the next Council meeting.
    \index{Minutes!Executive Meeting}
    \index{Executive, The!Executive Meetings!Minutes}

\end{longenum}
\newpage

\section{The Policy Manual}
\index{Policy Manual!Bylaw}
\begin{longenum}[ label*=\thesection.\arabic*., align=left]
	\item Policies arising from resolutions of Council or a general meeting that are not included in the Constitution of Bylaws shall be included in the Policy Manual.
    \index{Policy Manual!Codification of Council Motions}
    \index{Council!Policy Manual}
    \item Any amendments to the Policy Manual shall be accompanied by the date and origin of the amendment.
    \item The Policy Manual shall be publicly available in the same manner as other documents of the Society.
    \index{Policy Manual!Availability}
\end{longenum}
\newpage

\section{Communications}
\subsection{Communications}
\begin{longenum}[ label*=\thesection.\arabic*., align=left]
	\item The Executive will present all proposed statements of endorsement to Council for approval in principle. If Council grants approval, the Vice-President Advocacy, in consultation with the Executive, will draft and issue the statement on behalf of the Society. 
    \index{Vice-President Advocacy!Drafting and Issuing Statements}
    \index{Communications!Vice-President Advocacy}
    \item In time-sensitive situations where Council approval is not possible, the Executive shall be empowered to issue statements of endorsement on behalf of the Executive only; such statements must include a disclaimer clearly stating that they come from the Executive and not from the Society as a whole. 
    \index{Executive, The!Communications!Exigent Circumstances}
    \index{Communications!Vice-President Advocacy!Exigent Circumstances}
\end{longenum}
\newpage

\section{Non-Discrimination and Harassment}



\begin{longenum}[ label*=\thesection.\arabic*., align=left]
	\item Members of the Society and its employees are entitled to be free from discrimination and harassment. This may include but is not limited to discrimination or harassment on the basis of age, sex, gender, class, colour, national or ethnic origin, race, religion, creed, marital status, ability, sexual orientation, language, political belief medical condition(s), and criminal record or charges for which a pardon has been granted.
    \index{Non-Discrimination and Harassment!Protected Classes}
    \index{Non-Discrimination and Harassment!Definitions}


\item Activities considered to be harassment or discrimination may include, but are not limited to, adverse treatment, inappropriate gestures, remarks intended to intimidate or degrade, and exclusion based on discriminatory grounds.



\item The preference of the Society is to seek an informal resolution where ever possible. This may consist in direct discussion between the parties or, failing this, the parties may request the Ombudsperson to act as an informal mediator. (If the Ombudsperson cannot act as the mediator, a request may be made to the Speaker to act in their stead.)

   \index{Ombudsperson!Non-Discrimination and Harassment!Informal Resolution}
   \index{Non-Discrimination and Harassment!Informal Resolution}

\item If an informal resolution is not successful or possible, a formal complaint may be filed with the Ombudsperson for investigation and resolution as described in the Society's disciplinary manual.

   \index{Ombudsperson!Non-Discrimination and Harassment!Formal Resolution}
   \index{Non-Discrimination and Harassment!Formal Resolution}
\end{longenum}

\newpage

\section{Disciplinary Measures}
\subsection {General}
\begin{longenum}[ label*=\thesubsection.\arabic*., align=left] 
\item For all disciplinary matters which do not fall under Bylaw 11 (Conflict of Interest) the process shall be guided by the Society's Disciplinary Manual.
\end{longenum}
\subsection {Disciplinary Measures on an Employee} 
    \index{Discipline!Employees}

\begin{longenum}[ label*=\thesubsection.\arabic*., align=left]  
		\item Employees of the Society may face disciplinary measures described in the Personnel Manual, as described in Bylaw 10.
\end{longenum}
\subsection {List of Sanctions} 
As per section 5 of the Society's Constitution these are the totality of applicable sanctions that shall be given under the discipline manual in order of severity:
    \index{Discipline!Members!Sanctions}

\begin{longenum}[ label*=\thesubsection.\arabic*., align=left]
	\item Warnings
    \begin{longenum}[ label*=\arabic*., align=left]
		\item Verbal Warning from the President or other person in authority.
        \item Formal Censure.
	\end{longenum}
    \item Curtailment of rights and privileges:
    \begin{longenum}[ label*=\arabic*., align=left]
		\item A member may have their right to attend a Society sponsored or sanctioned event revoked, notwithstanding that granted by the Constitution of the Society.
        \item A member may have their right to enter the Grad Club revoked, notwithstanding that granted by the Constitution of the Society.
        \item A member may be prohibited from attending one or more Council meetings, notwithstanding the right granted by the Constitution of the Society and notwithstanding any ex-officio right to attend Council meetings.
	\end{longenum}
    \item Recall
    \begin{longenum}[ label*=\arabic*., align=left]
		\item A member holding the position of Executive or non-executive Officer may have a recall vote automatically initiated, without petition or motion of Council.
        \item In the case of an Executive who is not the President or in the case of a non-executive Officer, the Speaker shall place the recall vote on the agenda of the next scheduled Council meeting.
        \item In the case of the President, the Chief Returning Officer shall schedule a recall vote as if a valid petition had been received, as per Bylaw 7.1.1. In lieu of any ``Rationale'' text, the announcement of the recall vote shall include the text ``The recall of the President of the Society has been initiated by disciplinary measure and not by any petition'' followed by the full text of the written determination prompting the disciplinary measure. The announcement shall include the President's response to the written determination, if the President makes such a response. As there is no petitioner, the Chief Returning Officer shall assume the right to designate the petitioner's scrutineer, where appropriate.
        \item A Councillor may be automatically recalled, as if a petition had been submitted under Bylaw 7.1.1. The Speaker shall send a letter or email to the Councillor's duly constituted graduate student organization, or graduate secretary if such an organization does not exist, within three (3) business days informing them that the recall was due to disciplinary measures.
	\end{longenum}
    \item Limitations of prerogatives
    \begin{longenum}[ label*=\arabic*., align=left]
		\item A member may be prohibited from maintaining a particular ex-officio position, as described in Bylaws 1.2.4, 2.2.12, 2.5.5, 9.5.1.2, 9.5.1.3, 9.5.
        \item A member of the Executive may be prohibited from signing or endorsing contracts or cheques, as described in Bylaws 12.4.
	\end{longenum}
    \item Proscriptions
    \begin{longenum}[ label*=\arabic*., align=left]
		\item A member may be prohibited from holding any position on Society committees, Council, non--Executive and the Executive for a duration not exceeding twelve (12) months from the date the sanction is issued.
	\end{longenum}
\end{longenum}

\newpage

\section{Rules of Order}
\index{Rules of Order!Roberts}
\index{Roberts Rules of Order!Reserve Powers}
\begin{longenum}[ label*=\thesection.\arabic*., align=left]
	\item The rules contained in Robert's Rules of Order shall govern the Society in all cases to which they are applicable, and in which they are not inconsistent with the Constitution, Bylaws, and the Policies of the Society.
\end{longenum}
\newpage

\section{Dissemination of Amendments to the Society's Governing Documents}
\index{Bylaws and Constitution!Disseminating Changes}
\begin{longenum}[ label*=\thesection.\arabic*., align=left]
	\item For the purpose of this section, the Governing Documents consist of the Constitution, Bylaws and Policy Documents.
    \item Amendments shall enter Implementation Phase immediately upon ratification. 
    \item Enforcement of amendments during the Implementation Phase shall be limited to corrections and warnings. 
    \item Implementation Phase shall terminate in three (3) business days after the completion of both publication and notification of the membership. Amendments shall be considered fully actualized once the Implementation Phase has terminated.
    \item Council may choose to extend but not shorten the Implementation Phase. 
    \item The Speaker shall supervise the publication and notification of the amendments.
\end{longenum}
\newpage
\index{Bylaw!Bylaw Section|)}
